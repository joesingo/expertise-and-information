In this paper we present a modal logic framework to reason about the expertise
of information sources. A source is considered an expert on a proposition
$\phi$
if they are able to correctly refute $\phi$ in any possible world where $\phi$ is
false. Closely connected with expertise is a notion of \emph{soundness of
information}: $\phi$ is said to be ``sound" if it is true \emph{up to lack of
expertise} of the source. That is, any statement logically weaker than $\phi$
on which the source has expertise must in fact be true. This is relevant for
modelling situations in which sources make claims beyond their domain of
expertise.
%
Particular attention is paid to the connection between expertise and
\emph{knowledge}: we show that expertise and soundness admit precise
interpretations in terms of \emph{S4 and S5 epistemic logic}, under certain
conditions.
%
We go on to extend the framework to multiple sources, defining two notions of
\emph{collective expertise}. These also have epistemic interpretations via
distributed and common knowledge from multi-agent epistemic logic.
%
On the technical side, we give several sound and complete axiomatisations of
various classes of expertise models.
