\documentclass[12pt]{article}
\usepackage{xcolor}
\usepackage{amsmath}
\usepackage{amssymb}
\usepackage{quoting}

\title{%
    Responses to Review Comments
}
\author{}
\date{}

\newcommand\todo[1]{{\color{red} [\textbf{TODO:} {#1}]}}

\newenvironment{comment}{
    \noindent\textbf{Comment:}\
    \em
}{\vspace{5mm}}

\newenvironment{response}{
    \noindent\textbf{Response:}\
}{\vspace{5mm}}

\renewcommand{\phi}{\varphi}
\newcommand{\E}{\mathsf{E}}
\newcommand{\A}{\mathsf{A}}
\newcommand{\K}{\mathsf{K}}
\renewcommand{\S}{\mathsf{S}}

\begin{document}

\maketitle

Dear editor and reviewers,\\

We thank you for your detailed comments and suggestions for improving the
paper. The typos, minor issues regarding presentation and bibliographic errors
have been fixed in the revised paper. Please find more detailed responses to
each reviewer below.

\section*{Reviewer 1}

\begin{comment}
    In order to fully understand the intuitive idea of ``soundness'', I had to
    go to Singleton (2021). The way it’s phrased here, one would think that
    everything which entails a proposition about which the source is an expert
    would be sound. Of course, this would mean that reporting a contradiction
    like $p \land \neg p$ would be vacuously sound, which we see is not the
    case when the semantics is introduced.
\end{comment}

\begin{response}
    We agree with this comment, and have changed the informal descriptions of
    soundness in the abstract and introduction. Reviewer 2 also noted this
    issue and put forward a counterexample to this principle. We refer to the
    response to reviewer 2 for the extra assumptions one needs in order for
    this principle to hold, and the changes we have made to the text for
    clarification.
\end{response}

\begin{comment}
    I know what this sentence is driving at, but I don’t quite agree with it on
    an intuitive level. Say $A$ has a PhD on how the weather affects mental
    health (but she's not an expert on mental health broadly speaking), and $B$
    is a weatherman; between the two of them they do not make an expert on
    mental health, unless $A$ was an expert already.  I guess there is a subtle
    but important distinction between being an expert on the assertion ``$\phi$
    implies $\psi$’ and being an expert on the way in which $\phi$ affects
    $\psi$.
\end{comment}

\begin{response}
    On reflection we agree that this example does not work as initially
    intended. We have re-written the paragraph which introduces collective
    expertise, now referring to a specific case of group expertise from
    history instead of a hypothetical example.

    \begin{quotation}
        Beyond the individual expertise of a single source, one can also
        consider the \emph{collective expertise} of a group. For example, a
        committee may consist of several experts across different domains, so
        that by working together the group achieves expertise beyond any of its
        individual members. Indeed, such pooling of expertise becomes necessary
        in cases where it is infeasible for an individual to be a specialist in
        all relevant sub-areas. As a concrete example, consider the
        \emph{Rogers Commission report} into the 1986 Challenger disaster,
        whose members included politicians, military generals, physicists,
        astronauts and rocket scientists. Beyond extending the expertise of its
        constituents, the breadth of expertise among the commission allowed it
        to collectively assess issues at the \emph{intersection} of its
        members' specialities.
    \end{quotation}

\end{response}

\begin{comment}
    In general, I think this introduction could do a better job at motivating
    the paper, and could perhaps benefit from being a few lines longer
\end{comment}

\begin{response}
    In addition to changes described above, we have added an extra paragraph
    with some example settings in which reasoning about expertise is relevant,
    in an effort to better motivate the paper.

    \begin{quotation}
        Problems associated with expertise have been exacerbated recently by
        the COVID-19 pandemic, in which false information from non-experts has
        been shared widely on social media~(Van Dijck \& Alinejad, 2020;
        Llewellyn, 2020).  There have also been high-profile instances of
        experts going beyond their area of expertise to comment on issues of
        public health~(Xaudiera \& Cardenal, 2020), highlighting the importance
        of \emph{domain-specific} notions of expertise. Identifying experts is
        also an important task for \emph{liquid democracy}~(Blum \& Zuber,
        2016), in which voters may delegate their votes to expertise on a given
        policy issue.
    \end{quotation}
\end{response}

\begin{comment}
    Page 13: In 1-agent S5 models one can assume without loss of generality
    that the ``knowledge'' modality is universal, for the equivalence class
    partitions the model into the generated submodels given by the equivalence
    classes, each of them with a universal relation. Of course, if we attempt
    to do this in this framework we obtain that the $\A$ and the
    $\K$ modality have the same semantics; I would have loved to see
    some discussion on why this does not work in the present framework.
\end{comment}

\begin{response}
    The difference here is that we work with the language $\mathcal{L}_{\K\A}$,
    which includes not just the knowledge modality $\K$ but also the universal
    modality $\A$. Since in the relational semantics over this language
    $\A\phi$ is always interpreted with the universal relation $X \times X$,
    the notation of a generated submodel trivialises: the whole space $X$
    itself is the only set closed under the universal relation. For this
    reason, the submodels corresponding to the equivalence classes do not
    preserve the truth value of modal formulas, as would usually be the case
    with just the $\K$ modality, and thus the assumption that knowledge is
    universal is \emph{not} without loss of generality.
\end{response}

\begin{comment}
   Page 15: if I’m not mistaken, Corollary 1 follows not from the immediately
   previous Lemma 8 but from Lemma 7 plus the (EA) axiom: if $\E\phi \in
   \Sigma$ then (by (EA)) $\A\E\phi \in \Sigma$ which implies (by the
   definition of $R$) $\E\phi \in \Gamma$.
\end{comment}

\begin{response}
   This is absolutely correct for the second part of Corollary 1. One can also
   show the first part ($\A\phi \in \Gamma$ iff $\A\phi \in \Delta$) without
   Lemma 8 by additionally showing that $\mathbf{4}$ ($\A\phi \rightarrow
   \A\A\phi$) is provable in $\mathbf{KT5}$. While this derivation is fairly
   short, we do need Lemma 8 elsewhere in the paper (e.g. in the truth lemma,
   Lemma 9). For this reason (and to save limited space), we have kept
   Corollary 1 after Lemma 8.
\end{response}

\begin{comment}
    Page 21: since the $P_j$'s in this section are Alexandrov topologies, it
    would be nice to bring up the notion of topological join when defining
    $P^\mathsf{dist}_J$.
\end{comment}

\begin{response}
    This is a good point, and we have added a paragraph to note this connection
    after the introduction of $P^\mathsf{dist}_J$.

    \begin{quotation}
        $P^\mathsf{dist}_J$ also has a topological interpretation. As in
        Section 4, each $P_j$ gives rise to an Alexandrov topology $\tau_j$
        (where $P_j$ are the closed sets) if it is closed under unions and
        intersections. By the aforementioned properties, $\tau^\mathsf{dist}_J$
        corresponds to the coarsest Alexandrov topology finer than each
        $\tau_j$. On the other hand, since the join (in the lattice of
        topologies on $X$) of finitely many Alexandrov topologies is again
        Alexandrov~(Steiner, 1966, Theorems 2.4, 2.5), it follows that
        $\tau^\mathsf{dist}_J$ is equal to the join $\bigvee_{j \in
        J}{\tau_j}$.
    \end{quotation}

    Note that we have taken a slightly different interpretation to the
    reviewer's original suggestion, by viewing each $P_j$ as the \emph{closed}
    sets of a topology, i.e. $\tau_j = \{A \subseteq X \mid X \setminus A \in
    P_j\}$ (since $P_j$ is closed under unions and intersections, so too is
    $\tau_j$). This is to match our earlier topological interpretations in
    sections 2, 4 and 5, e.g. so that $\|\S\phi\|$ is the closure of
    $\|\phi\|$.

\end{response}

\section*{Reviewer 2}

\begin{comment}
     Also, I think that the title ``Expertise and knowledge: a modal logic
     perspective" is confusing in some sense. It may make one think of a logic
     on both expertise and knowledge, but the main contribution of the paper is
     logics just on expertise.
\end{comment}

\begin{response}
    We have now changed the title to ``Expertise and information: an epistemic
    logic perspective". We hope this eliminates the suggestion that we
    introduce new logics of knowledge, and instead refer to the existing
    literature on epistemic logic. We also mention ``information" in the
    title, since the core notion of \emph{soundness} in the paper is centred
    around non-expert information.
\end{response}

\begin{comment}
    My third worry is whether there is any inconsistency between the logical
    designs and the example used in Abstract, which states that ``if a source
    has expertise on $\phi$ but not $\psi$, when the conjunction $\phi \land
    \psi$ is sound whenever $\phi$ holds, since we can ignore $\psi$ (on which
    the source has no expertise)''. It looks that the example in Abstract aims
    to argue that the principle ``$\E\phi \land \neg\E\psi \land \phi
    \rightarrow \S(\phi \land \psi)$'' is valid, but it is not hard to
    construct a counterexample to it. This makes me doubt whether or not the
    logical design developed in the paper really captures the example in its
    motivation. Definitely, it may become valid if we restrict the class of
    models, but what the desired conditions should be?
\end{comment}

\begin{response}
    This is a good point, and indeed the principle extrapolated from the
    abstract is not in general valid. It seems the reviewer's model does not
    quite constitute a countermodel of $\E\phi \land \neg\E\psi \land \phi
    \rightarrow \S(\phi \land \psi)$, since it shows $M, 2 \models \E{p} \land
    \neg\E{q} \land p$ -- so that $\phi$ is $p$ and $\psi$ is $q$ -- but $M, 2
    \not\models \S(p \land \neg{q})$. Note that $q$ is negated, so this
    corresponds to $\S(\phi \land \neg\psi)$ instead of $\S(\phi \land \psi)$.
    %
    However, a minor modification yields a counterexample: taking $P = \{\{1,
    2\}, \{2\}\}$ instead, we have $M, 1 \models \E{p} \land \neg\E{q} \land p
    \land \neg\S(p \land q)$.

    As suggested, there are some further restrictions upon which the principle
    from the abstract becomes valid, e.g. if one additionally assumes expertise
    is closed under (finite) intersections, that the source does not have
    expertise on any non-empty set strictly stronger than $\phi$ (in the sense
    that $\emptyset \subset A \subseteq \|\phi\|$ implies $A = \|\phi\|$ for
    all $A \in P$) and that $\phi \land \psi$ is consistent. This is the case
    in Example 2, for instance.

    Since these additional assumptions are not suitable for describing the
    intuitive concept of soundness before the semantics are defined, we have
    removed the example from the abstract, replacing it with the following:

    \begin{quotation}
        Closely connected with expertise is a notion of \emph{soundness of
        information}: $\phi$ is said to be ``sound" if it is true \emph{up to
        lack of expertise} of the source. \textbf{That is, any statement
        logically weaker than $\phi$ on which the source has expertise must in
        fact be true. This is relevant for modelling situations in which
        sources make claims beyond their domain of expertise.}
    \end{quotation}

    We have also reworked the same example in the introduction:

    \begin{quotation}
        This formalises the idea of ``filtering out" parts of a statement
        within a source's expertise. For example, suppose $\phi = p \land q$,
        and the source has expertise on $p$ but not $q$. Supposing $p$ is true
        but $q$ is false, $\phi$ is false. However, if we discard information
        by ignoring $q$ (on which the source has no expertise), we obtain the
        weaker formula $p$, on which the source \emph{does} have expertise, and
        which is true. \textbf{If this holds for all possible ways to weaken $p
        \land q$ (this is the case, for instance, if the source does not have
        expertise on any statement strictly stronger than $p$), then $p \land
        q$ is \emph{false} but \emph{sound} for the source to report.}
    \end{quotation}

    We believe the revised versions better match the logical framework we go on
    to study.
\end{response}

\end{document}
